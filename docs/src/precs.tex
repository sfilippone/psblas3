\section{Preconditioner routines}
\label{sec:precs}

% \section{Preconditioners}
\label{sec:psprecs}
The base PSBLAS library  contains the implementation of two simple
preconditioning techniques:
\begin{itemize}
\item Diagonal Scaling
\item Block Jacobi with ILU(0) factorization
%% \item Additive Schwarz with the Restricted Additive Schwarz and
%%   Additive Schwarz with Harmonic extensions;
\end{itemize}
The supporting data type and subroutine interfaces are defined in the
module  \verb|psb_prec_mod|.
The old interfaces \verb|psb_precinit| and  \verb|psb_precbld| are still supported for
backward compatibility
%% We also provide a companion  package of multi-level Additive
%%   Schwarz preconditioners called MD2P4; this is actually a family of 
%%   preconditioners since there is the possibility to choose between
%%   many variants, and is currently in an experimental stateIts
%%   documentation is planned to appear after stabilization of the
%%   package, which will characterize release  2.1 of our library.



\clearpage\subsection{init --- Initialize a  preconditioner}

\begin{verbatim}
call prec%init(icontxt,ptype, info)
\end{verbatim}

\begin{description}
\item[Type:] Asynchronous.
\item[\bf On Entry]
\item[icontxt] the communication context.\\
Scope:{\bf global}.\\
Type:{\bf required}.\\
Intent: {\bf in}.\\
Specified as: an integer value.
\item[ptype] the type of preconditioner. 
Scope: {\bf global} \\
Type: {\bf required}\\
Intent: {\bf in}.\\
Specified as: a character string, see usage notes.
%% \item[iv] integer parameters for the precondtioner. 
%% Scope: {\bf global} \\
%% Type: {\bf required}\\
%% Specified as: an integer array, see usage notes. 
%% \item[rs] 
%% Scope: {\bf global} \\
%% Type: {\bf optional}\\
%% Specified as: a long precision real number.
\item[\bf On Exit]

\item[prec] 
Scope: {\bf local} \\
Type: {\bf required}\\
Intent: {\bf inout}.\\
Specified as: a preconditioner data structure \precdata.
\item[info] 
Scope: {\bf global} \\
Type: {\bf required}\\
Intent: {\bf out}.\\
Error code: if no error, 0 is returned.
\end{description}
{\par\noindent\large\bfseries Notes}
%% The PSBLAS 2.0 contains a number of preconditioners, ranging from a
%% simple diagonal scaling to 2-level domain decomposition. These
%% preconditioners may use the SuperLU or the UMFPACK software, if
%% installed; see~\cite{SUPERLU,UMFPACK}. 
Legal inputs to this subroutine are interpreted depending on the
$ptype$ string as follows\footnote{The string is case-insensitive}:
\begin{description}
\item[NONE] No preconditioning, i.e. the preconditioner is just a copy
  operator.
\item[DIAG] Diagonal scaling; each entry of the input vector is
  multiplied by the reciprocal of the sum of the absolute values of
  the coefficients in the corresponding row of matrix  $A$;
\item[BJAC] Precondition by a  factorization of the
  block-diagonal of matrix $A$, where block boundaries are determined
  by the data allocation boundaries for each process; requires no
  communication. Only the incomplete factorization $ILU(0)$ is
  currently implemented.  
\end{description}


\clearpage\subsection{build --- Builds a preconditioner}

\begin{verbatim}
call prec%build(a, desc_a, info[,amold,vmold,imold])
\end{verbatim}

\begin{description}
\item[Type:] Synchronous.
\item[\bf On Entry]
\item[a] the system sparse matrix.
Scope: {\bf local} \\
Type: {\bf required}\\
Intent: {\bf in}, target.\\
Specified as: a sparse matrix data structure \spdata.
\item[prec] the preconditioner.\\
Scope: {\bf local} \\
Type: {\bf required}\\
Intent: {\bf inout}.\\
Specified as: an already initialized precondtioner data structure \precdata\\
\item[desc\_a] the problem communication descriptor. 
Scope: {\bf local} \\
Type: {\bf required}\\
Intent: {\bf in}, target.\\
Specified as: a communication descriptor data structure \descdata.
%% \item[upd] 
%% Scope: {\bf global} \\
%% Type: {\bf optional}\\
%% Intent: {\bf in}.\\
%% Specified as: a character.
\item[amold] The desired dynamic type for the internal matrix storage.\\
Scope: {\bf local}.\\
Type: {\bf optional}.\\
Intent: {\bf in}.\\
Specified as: an object of a class derived from \spbasedata. 
\item[vmold] The desired dynamic type for the internal vector storage.\\
Scope: {\bf local}.\\
Type: {\bf optional}.\\
Intent: {\bf in}.\\
Specified as: an object of a class derived from \vbasedata. 
\item[imold] The desired dynamic type for the internal integer vector storage.\\
Scope: {\bf local}.\\
Type: {\bf optional}.\\
Intent: {\bf in}.\\
Specified as: an object of a class derived from (integer) \vbasedata. 
\end{description}

\begin{description}
\item[\bf On Return]
\item[prec] the preconditioner.\\
Scope: {\bf local} \\
Type: {\bf required}\\
Intent: {\bf inout}.\\
Specified as: a precondtioner data structure \precdata\\
\item[info] Error code.\\
Scope: {\bf local} \\
Type: {\bf required} \\
Intent: {\bf out}.\\
An integer value; 0 means no error has been detected. 
\end{description}
The \verb|amold|, \verb|vmold| and \verb|imold|  arguments may be
employed to interface with special devices, such as GPUs and other
accelerators. 



\clearpage\subsection{apply --- Preconditioner application
  routine}

\begin{verbatim}
call prec%apply(x,y,desc_a,info,trans,work)
call prec%apply(x,desc_a,info,trans)
\end{verbatim}

\begin{description}
\item[Type:] Synchronous.
\item[\bf On Entry]
\item[prec] the preconditioner.
Scope: {\bf local} \\
Type: {\bf required}\\
Intent: {\bf in}.\\
Specified as: a preconditioner data structure \precdata.
\item[x] the source vector.
Scope: {\bf local} \\
Type: {\bf required}\\
Intent: {\bf inout}.\\
Specified as:  a rank one array or an object of type \vdata. 
\item[desc\_a] the problem communication descriptor.
Scope: {\bf local} \\
Type: {\bf required}\\
Intent: {\bf in}.\\
Specified as: a communication data structure \descdata.
\item[trans] 
Scope: {\bf } \\
Type: {\bf optional}\\
Intent: {\bf in}.\\
Specified as: a character.
\item[work] an optional work space
Scope: {\bf local} \\
Type: {\bf optional}\\
Intent: {\bf inout}.\\
Specified as: a double precision array.
\end{description}

\begin{description}
\item[\bf On Return]
\item[y] the destination vector.
Scope: {\bf local} \\
Type: {\bf required}\\
Intent: {\bf inout}.\\
Specified as:  a rank one array or an object of type \vdata.
\item[info] Error code.\\
Scope: {\bf local} \\
Type: {\bf required} \\
Intent: {\bf out}.\\
An integer value; 0 means no error has been detected. 
\end{description}



\clearpage\subsection{descr --- Prints a description of current
  preconditioner}

\begin{verbatim}
call prec%descr()
call prec%descr(iout, root)
\end{verbatim}

\begin{description}
\item[Type:] Asynchronous.
\item[\bf On Entry]
\item[prec] the preconditioner.
Scope: {\bf local} \\
Type: {\bf required}\\
Intent: {\bf in}.\\
Specified as: a preconditioner data structure \precdata.
\item[iout] output unit.
Scope: {\bf local} \\
Type: {\bf optional}\\
Intent: {\bf in}.\\
Specified as: an integer number. Default: default output unit.
\item[root] Process from which to print
Scope: {\bf local} \\
Type: {\bf optional}\\
Intent: {\bf in}.\\
Specified as: an integer number between 0 and $np-1$, in which case
the specified process will print the description, or $-1$, in which case
all processes will print. Default: 0. 
\end{description}



\clearpage\subsection{clone --- clone  current
  preconditioner}
\begin{verbatim}
call  prec%clone(precout,info)
\end{verbatim}

\begin{description}
\item[Type:] Asynchronous.
\item[\bf On Entry]
\item[prec] the preconditioner.\\
Scope: {\bf local}.\\
% Type: {\bf required}.\\
% Intent: {\bf in}.\\
% Specified as: a object of type \descdata.
\end{description}

\begin{description}
\item[\bf On Return]
\item[precout] A copy of the input object.
\item[info] Return code. 
\end{description}


\clearpage\subsection{free --- Free a  preconditioner}

\begin{verbatim}
call prec%free(info)
\end{verbatim}

\begin{description}
\item[Type:] Asynchronous.
\item[\bf On Entry]
\item[prec] the preconditioner.\\
Scope: {\bf local}.\\
Type: {\bf required}\\
Intent: {\bf inout}.\\
Specified as: a preconditioner data structure \precdata.
%% \item[iv] integer parameters for the precondtioner. 
%% Scope: {\bf global} \\
%% Type: {\bf required}\\
%% Specified as: an integer array, see usage notes. 
%% \item[rs] 
%% Scope: {\bf global} \\
%% Type: {\bf optional}\\
%% Specified as: a long precision real number.
\item[\bf On Exit]

\item[prec] 
Scope: {\bf local} \\
Type: {\bf required}\\
Intent: {\bf inout}.\\
Specified as: a preconditioner data structure \precdata.
\item[info] 
Scope: {\bf global} \\
Type: {\bf required}\\
Intent: {\bf out}.\\
Error code: if no error, 0 is returned.
\end{description}
{\par\noindent\large\bfseries Notes}
%% The PSBLAS 2.0 contains a number of preconditioners, ranging from a
%% simple diagonal scaling to 2-level domain decomposition. These
%% preconditioners may use the SuperLU or the UMFPACK software, if
%% installed; see~\cite{SUPERLU,UMFPACK}. 
Releases all internal storage.

%%% Local Variables: 
%%% mode: latex
%%% TeX-master: "userguide"
%%% End: 
